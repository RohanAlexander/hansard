\documentclass[12pt,]{article}
\usepackage[left=1in,top=1in,right=1in,bottom=1in]{geometry}
\newcommand*{\authorfont}{\fontfamily{phv}\selectfont}
\usepackage[]{mathpazo}


  \usepackage[T1]{fontenc}
  \usepackage[utf8]{inputenc}



\usepackage{abstract}
\renewcommand{\abstractname}{}    % clear the title
\renewcommand{\absnamepos}{empty} % originally center

\renewenvironment{abstract}
 {{%
    \setlength{\leftmargin}{0mm}
    \setlength{\rightmargin}{\leftmargin}%
  }%
  \relax}
 {\endlist}

\makeatletter
\def\@maketitle{%
  \newpage
%  \null
%  \vskip 2em%
%  \begin{center}%
  \let \footnote \thanks
    {\fontsize{18}{20}\selectfont\raggedright  \setlength{\parindent}{0pt} \@title \par}%
}
%\fi
\makeatother




\setcounter{secnumdepth}{0}

\usepackage{longtable,booktabs}

\usepackage{graphicx,grffile}
\makeatletter
\def\maxwidth{\ifdim\Gin@nat@width>\linewidth\linewidth\else\Gin@nat@width\fi}
\def\maxheight{\ifdim\Gin@nat@height>\textheight\textheight\else\Gin@nat@height\fi}
\makeatother
% Scale images if necessary, so that they will not overflow the page
% margins by default, and it is still possible to overwrite the defaults
% using explicit options in \includegraphics[width, height, ...]{}
\setkeys{Gin}{width=\maxwidth,height=\maxheight,keepaspectratio}

\title{Nobody expects the Spanish Inquisition: Which events drive topic change
in Australian parliaments? \thanks{Thank you to John Tang, Zach Ward, Tim Hatton and Martine Mariotti for
their helpful comments, guidance and suggestions. \textbf{Version as
of}: September 15, 2018; \textbf{Comments welcome}:
\href{mailto:rohan.alexander@anu.edu.au}{\nolinkurl{rohan.alexander@anu.edu.au}}}  }



\author{\Large Monica Alexander\vspace{0.05in} \newline\normalsize\emph{University of Toronto}   \and \Large Rohan Alexander\vspace{0.05in} \newline\normalsize\emph{Australian National University}  }


\date{}

\usepackage{titlesec}

\titleformat*{\section}{\normalsize\bfseries}
\titleformat*{\subsection}{\normalsize\itshape}
\titleformat*{\subsubsection}{\normalsize\itshape}
\titleformat*{\paragraph}{\normalsize\itshape}
\titleformat*{\subparagraph}{\normalsize\itshape}


\usepackage{natbib}
\bibliographystyle{apsr}
\usepackage[strings]{underscore} % protect underscores in most circumstances



\newtheorem{hypothesis}{Hypothesis}
\usepackage{setspace}

\makeatletter
\@ifpackageloaded{hyperref}{}{%
\ifxetex
  \PassOptionsToPackage{hyphens}{url}\usepackage[setpagesize=false, % page size defined by xetex
              unicode=false, % unicode breaks when used with xetex
              xetex]{hyperref}
\else
  \PassOptionsToPackage{hyphens}{url}\usepackage[unicode=true]{hyperref}
\fi
}

\@ifpackageloaded{color}{
    \PassOptionsToPackage{usenames,dvipsnames}{color}
}{%
    \usepackage[usenames,dvipsnames]{color}
}
\makeatother
\hypersetup{breaklinks=true,
            bookmarks=true,
            pdfauthor={Monica Alexander (University of Toronto) and Rohan Alexander (Australian National University)},
             pdfkeywords = {text analysis, politics, Australia},  
            pdftitle={Nobody expects the Spanish Inquisition: Which events drive topic change
in Australian parliaments?},
            colorlinks=true,
            citecolor=blue,
            urlcolor=blue,
            linkcolor=magenta,
            pdfborder={0 0 0}}
\urlstyle{same}  % don't use monospace font for urls

% set default figure placement to htbp
\makeatletter
\def\fps@figure{htbp}
\makeatother



% add tightlist ----------
\providecommand{\tightlist}{%
\setlength{\itemsep}{0pt}\setlength{\parskip}{0pt}}

\begin{document}
	
% \pagenumbering{arabic}% resets `page` counter to 1 
%
% \maketitle

{% \usefont{T1}{pnc}{m}{n}
\setlength{\parindent}{0pt}
\thispagestyle{plain}
{\fontsize{18}{20}\selectfont\raggedright 
\maketitle  % title \par  

}

{
   \vskip 13.5pt\relax \normalsize\fontsize{11}{12} 
\textbf{\authorfont Monica Alexander} \hskip 15pt \emph{\small University of Toronto}   \par \textbf{\authorfont Rohan Alexander} \hskip 15pt \emph{\small Australian National University}   

}

}








\begin{abstract}

    \hbox{\vrule height .2pt width 39.14pc}

    \vskip 8.5pt % \small 

\noindent We use a structural text model to explore the effect of various events
on what was said in Australian state and federal parliaments from the
mid-1800s through to 2017. We find that: 1) changes of government are
associated with topic changes only when there is also a change in the
party in power; 2) polling results appear dissociated from parliamentary
topics; 3) economic changes, such as financial crises have a significant
effect; and 4) other events, such as an unexpected attack tend not to
have a prolonged change.


\vskip 8.5pt \noindent \emph{Keywords}: text analysis, politics, Australia \par

    \hbox{\vrule height .2pt width 39.14pc}



\end{abstract}


\vskip 6.5pt


\noindent  \section{Introduction}\label{introduction}

New governments often go to some trouble to be different from the
governments they replace. For instance, Kevin Rudd's apology to
Indigenous Australians was not supported by John Howard, and one of Tony
Abbott's first acts was to repeal Rudd's carbon tax. Similarly,
significant events can alter the course of a government. For instance,
consider the change in the Howard government after the 9/11 attacks or
the 2002 Bali bombings. However it is not so clear which events drive
changes in topics, for instance, do they change when the government is
replaced by another of its own party? And which events are temporary,
for instance when an economic crisis abates, do the topics return to
pre-crisis levels?

In this paper we use the Structural Topic Model (STM) of
\citet{RobertsStewartAiroldi2016} to model the topics of discussion in
Australian parliaments. The advantage of this model is that it allows
for topics to be correlated between sitting days, which then allows us
to test for changes in topics at various events. The events that we
focus on are changes in: 1) government; 2) the political environment (as
defined by polling or other results); 3) economic conditions; and 4) the
significant events (such as the 9/11 attacks or the Bali Bombings).

We find {[}INSERT RESULTS{]}. We also explored the other direction (the
impact of what was said in parliaments on events) but it was difficult
to find significant effects.

Our work fits into {[}WHATEVER IT FITS INTO{]}. While using text as data
has well-known shortcomings, it allows larger-scale analysis that would
not be viable using less-automated approaches and hence can identify
patterns that may otherwise be overlooked.

\section{Data}\label{data}

Following the example of the UK a text record called Hansard of what was
said in Australian parliaments has been made available since their
establishment.\footnote{While Hansard is not necessarily verbatim, it is
  considered close enough for text-as-data purposes. For instance,
  \citet{Mollin2008} found that in the case of the UK Hansard the
  differences would only affect specialised linguistic analysis.
  \citet{Edwards2016} examined Australia, New Zealand and the UK, and
  found that changes were usually made by those responsible for creating
  the Hansard record, instead of the parliamentarians. Both these
  findings provide reassurance that differences between Hansard and a
  verbatim record would not be meaningful for this paper.} Hansard
records are an increasingly popular source of data as new methods and
reduced computational costs make larger-scale analysis easier. For
instance, the digitisation of the Canadian Hansard,
\citet{BeelenEtc2017}, allowed \citet{Whyte2017} to examine whether
parliamentary disruptions in Canada increased between 1926 and 2015. In
the UK, \citet{Duthie2016} analysed Hansard records to examine which
politicians made supportive or aggressive statements toward other
politicians between 1979 and 1990, and \citet{Willis2017} examined how
politicians understood climate change. In New Zealand,
\citet{Curran2017} modelled the topics discussed between 2003 and 2016,
and \citet{Graham2016} examined unparliamentary language between 1890
and 1950.

Australian Hansard records have been analysed for various purposes, but
usually not at scale. For instance, \citet{Rasiah2010} examines Hansard
records for the Australian House of Representatives to examine whether
politicians attempted to evade questions about Iraq during February and
March 2003. \citet{GansLeigh2012} examined Australian Hansard records by
hand to associate mentions by politicians of certain public
intellectuals with neutral or positive sentiment.

The Australian parliaments generally make their Hansard records
available online as PDFs that can be downloaded. The Federal parliament
additionally makes XML records available for years between 1901 and 1980
as well as from 1997. There are roughly 65,000 \textbf{(UPDATE)} hansard
records available across the chambers of the state and federal
parliaments (Table 1) \textbf{(UPDATE NUMBERING)}. As with any
larger-scale data process, there are many issues with this dataset of
PDFs and the known ones are detailed in the Appendix.

\begin{longtable}[]{@{}lrrr@{}}
\toprule
Parliament & House & Years used & Notes\tabularnewline
\midrule
\endhead
Commonwealth & House of Representatives & 1901 - 2017 & -\tabularnewline
Queensland & ? & 1861 - 2017 & -\tabularnewline
New South Wales & ? & ? - 2017 & -\tabularnewline
Victoria & ? & ? - 2017 & -\tabularnewline
Tasmania & ? & ? - 2017 & -\tabularnewline
South Australia & ? & ? - 2017 & -\tabularnewline
Western Australia & ? & ? - 2017 & -\tabularnewline
\bottomrule
\end{longtable}

The PDFs were processed using the \texttt{PDFtools} R package of
\citet{Ooms2018pdftools}. A small proportion of the Hansard records had
not been put through a professional OCR process and although the Google
Tesseract engine as implemented by \citet{Ooms2018tesseract} provided
some useful data, these were not used in this analysis.

to extract these as text-based CSV records, and clean the records using
functions from the \texttt{Tidyverse} R package of
\citet{WickhamHadleyTidyverse}, the \texttt{Tidytext} R package of
\citet{SilgeRobinson2016} and the {[}INSERT OTHERS{]}. The result is a

\section{Model}\label{model}

The primary model that we use in this paper is the Structural Topic
Model (STM) as implemented by the \texttt{STM} R package of
\citet{RobertsStewartAiroldiRPackage}. The basis of this type of topic
modelling is the Latent Dirichlet Allocation (LDA) model of
\citet{Blei2003latent}. In this section a brief overview of both the LDA
model and the STM approach is provided and then the specifics of how we
consider events in this setting are discussed.

\subsection{Latent Dirichlet
Allocation}\label{latent-dirichlet-allocation}

Each day's Hansard record needs to be classified by its topic. Sometimes
Hansard includes titles that make the topic clear. But not every
statement has a title and the titles do not always define topics in a
well-defined and consistent way, especially over longer time periods.
One way to get consistent estimates of the topics of each statement in
Hansard is to use the latent Dirichlet allocation (LDA) method of
\citet{Blei2003latent}, for instance as implemented by the R package
`topicmodels' by \citet{Grun2011}.

The key assumption behind the LDA method is that each day's text, `a
document', in Hansard is made by speakers who decide the topics they
would like to talk about in that document, and then chooses words,
`terms', that are appropriate to those topics. A topic could be thought
of as a collection of terms, and a document as a collection of topics.
The topics are not specified \emph{ex ante}; they are an outcome of the
method. Terms are not necessarily unique to a particular topic, and a
document could be about more than one topic. This provides more
flexibility than other approaches such as a strict word count method.
The goal is to have the words found in each day's Hansard group
themselves to define topics.

As applied to Hansard, the LDA method considers each statement to be a
result of a process where a politician first chooses the topics they
want to speak about. After choosing the topics, the speaker then chooses
appropriate words to use for each of those topics.

More generally, the LDA topic model works by considering each document
as having been generated by some probability distribution over topics.
For instance, if there were five topics and two documents, then the
first document may be comprised mostly of the first few topics; the
other document may be mostly about the final few topics (Figure
@ref(fig:topicsoverdocuments)).

\includegraphics{svm-rmarkdown-article-example_files/figure-latex/topicsoverdocuments-1.pdf}
\includegraphics{svm-rmarkdown-article-example_files/figure-latex/topicsoverdocuments-2.pdf}

Similarly, each topic could be considered a probability distribution
over terms. To choose the terms used in each document the speaker picks
terms from each topic in the appropriate proportion. For instance, if
there were ten terms, then one topic could be defined by giving more
weight to terms related to immigration; and some other topic may give
more weight to terms related to the economy (Figure
@ref(fig:topicsoverterms)).

\begin{figure}
\includegraphics[width=.49\linewidth]{svm-rmarkdown-article-example_files/figure-latex/topicsoverterms-1} \includegraphics[width=.49\linewidth]{svm-rmarkdown-article-example_files/figure-latex/topicsoverterms-2} \caption{Probability distributions over terms}\label{fig:topicsoverterms}
\end{figure}

Following \citet{BleiLafferty2009}, \citet{blei2012} and
\citet{GriffithsSteyvers2004}, the process by which a document is
generated is more formally considered to be:

\begin{enumerate}
\def\labelenumi{\arabic{enumi}.}
\tightlist
\item
  There are \(1, 2, \dots, k, \dots, K\) topics and the vocabulary
  consists of \(1, 2, \dots, V\) terms. For each topic, decide the terms
  that the topic uses by randomly drawing distributions over the terms.
  The distribution over the terms for the \(k\)th topic is \(\beta_k\).
  Typically a topic would be a small number of terms and so the
  Dirichlet distribution with hyperparameter \(0<\eta<1\) is used:
  \(\beta_k \sim \mbox{Dirichlet}(\eta)\).\footnote{The Dirichlet
    distribution is a variation of the beta distribution that is
    commonly used as a prior for categorical and multinomial variables.
    If there are just two categories, then the Dirichlet and the beta
    distributions are the same. In the special case of a symmetric
    Dirichlet distribution, \(\eta=1\), it is equivalent to a uniform
    distribution. If \(\eta<1\), then the distribution is sparse and
    concentrated on a smaller number of the values, and this number
    decreases as \(\eta\) decreases. A hyperparameter is a parameter of
    a prior distribution.} Strictly, \(\eta\) is actually a vector of
  hyperparameters, one for each \(K\), but in practice they all tend to
  be the same value.
\item
  Decide the topics that each document will cover by randomly drawing
  distributions over the \(K\) topics for each of the
  \(1, 2, \dots, d, \dots, D\) documents. The topic distributions for
  the \(d\)th document are \(\theta_d\), and \(\theta_{d,k}\) is the
  topic distribution for topic \(k\) in document \(d\). Again, the
  Dirichlet distribution with the hyperparameter \(0<\alpha<1\) is used
  here because usually a document would only cover a handful of topics:
  \(\theta_d \sim \mbox{Dirichlet}(\alpha)\). Again, strictly \(\alpha\)
  is vector of length \(K\) of hyperparameters, but in practice each is
  usually the same value.
\item
  If there are \(1, 2, \dots, n, \dots, N\) terms in the \(d\)th
  document, then to choose the \(n\)th term, \(w_{d, n}\):

  \begin{enumerate}
  \def\labelenumii{\alph{enumii}.}
  \tightlist
  \item
    Randomly choose a topic for that term \(n\), in that document \(d\),
    \(z_{d,n}\), from the multinomial distribution over topics in that
    document, \(z_{d,n} \sim \mbox{Multinomial}(\theta_d)\).
  \item
    Randomly choose a term from the relevant multinomial distribution
    over the terms for that topic,
    \(w_{d,n} \sim \mbox{Multinomial}(\beta_{z_{d,n}})\).
  \end{enumerate}
\end{enumerate}

Given this set-up, the joint distribution for the variables is
(\citet{blei2012}, p.6):
\[p(\beta_{1:K}, \theta_{1:D}, z_{1:D, 1:N}, w_{1:D, 1:N}) = \prod^{K}_{i=1}p(\beta_i) \prod^{D}_{d=1}p(\theta_d) \left(\prod^N_{n=1}p(z_{d,n}|\theta_d)p\left(w_{d,n}|\beta_{1:K},z_{d,n}\right) \right).\]

Based on this document generation process the analysis problem,
discussed next, is to compute a posterior over \(\beta_{1:K}\) and
\(\theta_{1:D}\), given \(w_{1:D, 1:N}\). This is intractable directly,
but can be approximated (\citet{GriffithsSteyvers2004} and
\citet{blei2012}).

After the documents are created, they are all that we have to analyse.
The term usage in each document, \(w_{1:D, 1:N}\), is observed, but the
topics are hidden, or `latent'. We do not know the topics of each
document, nor how terms defined the topics. That is, we do not know the
probability distributions of Figures @ref(fig:topicsoverdocuments) or
@ref(fig:topicsoverterms). In a sense we are trying to reverse the
document generation process -- we have the terms and we would like to
discover the topics.

If the earlier process around how the documents were generated is
assumed and we observe the terms in each document, then we can obtain
estimates of the topics (\citet{SteyversGriffiths2006}). The outcomes of
the LDA process are probability distributions and these define the
topics. Each term will be given a probability of being a member of a
particular topic, and each document will be given a probability of being
about a particular topic. That is, we are trying to calculate the
posterior distribution of the topics given the terms observed in each
document (\citet{blei2012}, p.~7):
\[p(\beta_{1:K}, \theta_{1:D}, z_{1:D, 1:N} | w_{1:D, 1:N}) = \frac{p\left(\beta_{1:K}, \theta_{1:D}, z_{1:D, 1:N}, w_{1:D, 1:N}\right)}{p(w_{1:D, 1:N})}.\]

The initial practical step when implementing LDA given a collection of
documents is to remove `stop words'. These are words that are common,
but that don't typically help to define topics. There is a general list
of stop words such as: ``a''; ``a's''; ``able''; ``about'';
``above''\ldots{} An additional list of words that are commonly found in
Hansard, but likely don't help define topics is added to the general
list. These additions include words such as: ``act''; ``amendment'';
``amount''; ``australia''; ``australian''; ``bill''\ldots{} A full list
can be found in Appendix @ref(hansard-stop-word). We also remove
punctuation and capitalisation. The documents need to then be
transformed into a document-term-matrix. This is essentially a table
with a column of the number of times each term appears in each document.

After the dataset is ready, the R package `topicmodels' by
\citet{Grun2011} can be used to implement LDA and approximate the
posterior. It does this using Gibbs sampling or the variational
expectation-maximization algorithm. Following
\citet{SteyversGriffiths2006} and \citet{Darling2011}, the Gibbs
sampling process attempts to find a topic for a particular term in a
particular document, given the topics of all other terms for all other
documents. Broadly, it does this by first assigning every term in every
document to a random topic, specified by Dirichlet priors with
\(\alpha = \frac{50}{K}\) and \(\eta = 0.1\)
(\citet{SteyversGriffiths2006} recommends \(\eta = 0.01\)), where
\(\alpha\) refers to the distribution over topics and \(\eta\) refers to
the distribution over terms (\citet{Grun2011}, p.~7). It then selects a
particular term in a particular document and assigns it to a new topic
based on the conditional distribution where the topics for all other
terms in all documents are taken as given (\citet{Grun2011}, p.~6):
\[p(z_{d, n}=k | w_{1:D, 1:N}, z'_{d, n}) \propto \frac{\lambda'_{n\rightarrow k}+\eta}{\lambda'_{.\rightarrow k}+V\eta} \frac{\lambda'^{(d)}_{n\rightarrow k}+\alpha}{\lambda'^{(d)}_{-i}+K\alpha} \]
where \(z'_{d, n}\) refers to all other topic assignments;
\(\lambda'_{n\rightarrow k}\) is a count of how many other times that
term has been assigned to topic \(k\); \(\lambda'_{.\rightarrow k}\) is
a count of how many other times that any term has been assigned to topic
\(k\); \(\lambda'^{(d)}_{n\rightarrow k}\) is a count of how many other
times that term has been assigned to topic \(k\) in that particular
document; and \(\lambda'^{(d)}_{-i}\) is a count of how many other times
that term has been assigned in that document. Once \(z_{d,n}\) has been
estimated, then estimates for the distribution of words into topics and
topics into documents can be backed out.

This conditional distribution assigns topics depending on how often a
term has been assigned to that topic previously, and how common the
topic is in that document (\citet{SteyversGriffiths2006}). The initial
random allocation of topics means that the results of early passes
through the corpus of document are poor, but given enough time the
algorithm converges to an appropriate estimate.

The choice of the number of topics, \emph{k}, affects the results, and
must be specified \emph{a priori}. If there is a strong reason for a
particular number, then this can be used. Otherwise, one way to choose
an appropriate number is to use a test and training set process.
Essentially, this means running the process on a variety of possible
values for \emph{k} and then picking an appropriate value that performs
well.

One weakness of the LDA method is that it considers a `bag of words'
where the order of those words does not matter (\citet{blei2012}). It is
possible to extend the model to reduce the impact of the bag-of-words
assumption and add conditionality to word order. Additionally,
alternatives to the Dirichlet distribution can be used to extend the
model to allow for correlation. For instance, in Hansard topics related
the army may be expected to be more commonly found with topics related
to the navy, but less commonly with topics related to banking. This
motivates the use of the Structural Topic Model, described in the next
section.

\subsection{Structural Topic Model}\label{structural-topic-model}

\subsubsection{Overview and example}\label{overview-and-example}

{[}TBD{]}

Note that each of the states and the Commonwealth are treated
independently here. Future work could expand the model to better
understand, and allow, for correlation between them.

\subsubsection{Considering events}\label{considering-events}

{[}TBD{]}

\section{Results}\label{results}

\subsection{Political events}\label{political-events}

\begin{quote}
\emph{When you change the government, you change the country.} Paul
Keating.
\end{quote}

Change of government.

\subsection{Polling events}\label{polling-events}

\begin{quote}
\emph{The only poll that matters is the one on election day.} John
Howard.
\end{quote}

{[}TBD{]}

\subsection{Economic events}\label{economic-events}

Major economic changes.

{[}TBD{]}

\subsection{External events}\label{external-events}

\begin{quote}
\emph{Events, dear boy, events.} Attributed to Harold Macmillan.
\end{quote}

Major event such as 9/11 attacks, or economic change.

\section{Summary and conclusions}\label{summary-and-conclusions}

What could happen if we had longer terms. Eg GST needed multiple
generations of politicians but carbon tax couldn't because it was one
generation.

Text analysis has well-known biases and weaknesses and is a complement
to more detailed analysis such as qualitative methods and case studies.
We consider the results presented in this paper, as well as many of
those results of the larger text-as-data research program, as fitting
within findings based on other methods.

\newpage

\section{Appendix}\label{appendix}

\subsection{Document sources}\label{document-sources}

Where from?

Which years are being used (not non-OCRd)

\subsection{Dataset issues}\label{dataset-issues}

Which PDFs are missing or have no content, etc.

\subsection{PDF to CSV issues}\label{pdf-to-csv-issues}

Insert graph of stop words over time.

\subsection{Selection of number of
topics}\label{selection-of-number-of-topics}

\subsection{Robustness of results}\label{robustness-of-results}

Here we change the number of sitting days considered either side of an
event. The results in the main section of the paper are for the nearest
ten days either side of an event. Here are show that the results are
essentially the same if the nearest one, two, five, and twenty days
either side of an event.

\newpage




\newpage
\singlespacing 
\renewcommand\refname{References}
\bibliography{../bibliography.bib}

\end{document}
